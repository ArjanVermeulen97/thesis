\documentclass[a4paper,10pt]{article}
\usepackage[utf8]{inputenc}
\usepackage{amsmath}
\usepackage{graphicx}


%opening
\title{WP Mission and Systems Engineering}
\author{J. G. P. Vermeulen}

\begin{document}

\maketitle

\section{Mission}
From the literature study, we know the problem at hand: for a long time, 140 meters was seen as the diameter to design asteroid surveys for. The result is that a lot of smaller sized asteroids are still unknown. Survey completeness for asteroids such as the one that impacted Chelyabinsk in 2013 is only around 10\%. Several studies have been conducted on possible future space-based surveys, but the survey completeness they reach is not very impressive. 
\subsection{Determination of Mission}
\subsubsection{Mission Objective}
The main objective of the mission is to raise the completeness of Near-Earth Asteroid catalogues to the highest feasible level using technology that is currently available. A secondary objective of the system is to raise the warning time for asteroids that do end up impacting Earth.
\subsubsection{Mission Constraints}
The design of the system will be mostly constrained by the capabilities of the hardware. Survey telescopes can easily generate terabytes of data daily, and it is not feasible to send all the images back to Earth. Therefore, the system should be capable of processing the data itself, with only occassional intervention from Earth. Furthermore, we are constrained by the capabilities of present-day hardware in doing this: mainly the cameras, telecommunications, adcs and data processing systems. They will be further analysed in the next section
\subsubsection{Mission Statement}


\section{Systems Engineering Analysis}
\subsection{System Description}
The most promising options found by Stokes et al (2017) were 0.5m aperture telescopes operating in either the visual wavelength or the thermal infrared. On the one hand, visual sensors benefit from reduced temperature requirements, and larger fields of view, on the other hand, TIR sensors benefit from lower background signal and the fact that asteroids emit far more IR radiation than visual photons. In addition, TIR is not dependent on the object's albedo. From spacecraft doing similar missions (NEOWISE, NEOSSat, Akari, Spitzer), we can estimate the approximate launch mass of the satellites carrying the telescopes to be around 1000 kg. This is important for determining the capabilities of the system in terms of processing power, communications and ADCS. 
\subsubsection{Interaction between Spacecraft}

\subsubsection{Duty Cycles and Lifespan}

\subsubsection{Requirements}


\subsection{Communication}
\subsubsection{Link Budget}
https://en.wikipedia.org/wiki/NEO_Surveyor#cite_note-Mainzer2009sbag-31 gives a datarate of 82Gbit/day for the NEOCAM mission. We can deduce from this that transmitting full raw images takes approximately 1Mbit/s. Potentially, according to 2017 NEO SDT paper, we could drop the data rate significantly by only transmitting regions of interest, however this is unproven in flight and will require additional processing to be done on each of the spacecraft.
\subsubsection{Range}

\subsection{ADCS}
\subsubsection{Pointing Accuracy and Stability}
\subsubsection{Error Propagation among Multiple Satellites}

\subsection{Data Processing}
\subsubsection{Available Computing Power}
\subsubsection{Imposed Limitations}

\subsection{Payload}
\subsubsection{Relevant values}
\end{document}
