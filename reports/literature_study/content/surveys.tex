\chapter{Near-Earth Asteroid Surveys}
\label{ch:surveys}
Presently, an not inconsiderable amount of effort is already being spent on cataloguing NEA's (\cite{NEAsurveys}). Therefore, to determine the feasibility of the proposed system, it is important to assess the current capabilities of detection techniques, and to determine where the weaknesses of these systems lie. Firstly, an overview of Earth-based surveys will be given, followed by an overview of space-based surveys. This is by no means an exhaustive list, but these are the most prominent of the surveys

\section{Earth-based Surveys}

By far the majority of current and past NEA surveys have been carried out from Earth. Although there are disadvantages because of Solar glare, the interference of the atmosphere, the fixed orbit of Earth, and effects such as weather and moon phases, logistical reasons often outweigh these.

\subsection{Catalina Sky Survey}
The Catalina Sky Survey of the University of Arizona's Lunar and Planetary lab operates a $1.5$m telescope on Mount Lemmon in Arizona. This telescope is covering approximately $1000 \mathrm{deg}^2$ per night with a limiting magnitude of 21.5. In addition, the survey operates two smaller telescopes. The Catalina Sky Survey is based on a limiting size of $140$m, per the definition of a potentially hazardous asteroid, and has discovered over 12000 NEA's since the start of operation in 1995 (\cite{catalinaskysurvey}).

\subsection{Pan-STARRS}
Pan-STARRS, the Panoramic Survey Telescope and Rapid Response System features two $1.8$m telescopes at the Institute for Astronomy at the University of Hawaii. Their limiting magnitude is slightly better than the Catalina Sky Survey, at a theoretical maximum of 24. It has been operational since 2010 and is capable of imaging approximately $6000 \mathrm{deg}^2$ of sky per night. It is expected to discover a very large number of undiscovered NEA's, among other discoveries. For example, Pan-STARRS was also the telescope which discovered the 2017 interstellar object 'Oumuamua (\cite{Panstarrs}).

\subsection{ATLAS}
\label{ssec:ATLAS}
ATLAS, the Asteroid Terrestrial-impact Last Alert System designed around the idea of being a last alert system for asteroids approaching Earth. It "will provide one day's warning for a 30-kiloton 'town killer,' a week for a 5-megaton 'city killer,' and 3 weeks for a 100-megaton 'country killer'." (\cite{Atlasite}) The idea here is to provide a last alert for any asteroids missed by other surveys and allow humans time to evacuate. The two 0.5m telescopes of ATLAS are situated on two mountain peaks in Hawaii, and the observatories are run by the University of Hawaii. The impressive aspect of ATLAS is its high cadence: With the two telescopes working together, ATLAS is capable of imaging the night sky as visible from Hawaii four times every two days. This high cadence is required as multiple repeat observations are needed to compute orbits accurately. The team is looking to build more "ATLAS units" worldwide, although issues with weather and solar interference remain (\cite{ATLAS}).

\subsection{Zwicky Transient Facility}
\label{ssec:palomar}
While not a dedicated NEA facility, the Zwicky Transient Facility (sometimes referred to by the name of its predecessor, the Palomar Transient Factory ), yields many observations of NEA's using its 48-inch (approx 1.2 m) telescope. While not particularly powerful in terms of limiting magnitude, the Zwicky Transient Facility shines in its high survey cadence, capable of imaging the night sky every three nights, and the plane of the Milky Way twice per night. The ZTF is the precursor to the upcoming Large Synoptic Survey Telescope, which will feature an 8.4 meter telescope imaging the sky at a high cadence. It is expected to vastly increase the number of mapped objects (by a factor of 10-100), although this is not its primary goal (\cite{LSST}).

\section{Space-based Surveys}

Contrary to Earth-based systems, a space-based telescope can be used to image without interference from weather, or the atmosphere in general. It therefore provides superior optical performance, allowing smaller systems to be used effectively.

\subsection{NEOWISE}
NEOWISE was the second mission of the WISE spacecraft. Originally intended to image to entire sky in infrared using cooled optics, the spacecraft was placed in hibernation after the coolant ran out. In 2013, the spacecraft was woken from hibernation and assigned a new mission: to image the solar system for near-Earth objects. From December 2013 through 2017, NEOWISE detected 34000 new objects, among which 262 NEA's, of which 47 classified as a PHA (\cite{NEOWise}).

\subsection{NEOSSat}
\label{ssec:NEOSSat}
Launched in 2013, The Near-Earth Object Surveillance Satellite is a Canadian mission specifically scanning the sky near the Sun for inner-Earth objects, particularly the Aten and Atira classes. Their orbit makes them harder to detect from Earth, making a space-based survey ideal. NEOSSat is an interesting design when contrasted to the other surveys, as it only uses a 0.15m telescope mounted on a microsattelite, and uses its primary telescope also as a star sensor for attitude determination. However, despite these limitations, it is capable of imaging down to apparent magnitude 19 (\cite{Neossat}; \cite{Neossattwo}). 

\begin{figure}[hbtp]
    \centering
    \includegraphics[width=1.0\textwidth]{images/neossat.jpg}
    \caption{Artist rendition of the NEOSSat satellite, showing the apeture of the 0.15m Maksutov-Cassegrain telescope. (Image Courtesy: Canadian Space Agency)}
    \label{fig:neossat}
\end{figure}