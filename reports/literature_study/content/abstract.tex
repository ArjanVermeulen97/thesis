\chapter{Abstract}
\label{ch:abstract}

Ever since the Tunguska event of 1908 and the discovery linking the extinction of the non-avian dinosaurs to the impact of a meteor, humanity has been searching the skies in an attempt to discover asteroids on an impact trajectory before their collision with Earth. In this literature review, comprehensive background research is provided on proposed research contributing to the problem of impact-hazardous asteroid identification. \\

Through discussion of the current knowledge on the near-Earth asteroid population, optics, and currently existing asteroid surveys, an autonomous space-based survey system is proposed. This system has to be capable of highly autonomously assessing the risk posed by objects it detects. Then, the state-of-the-art in image processing and trajectory determination techniques is discussed: firstly the image has to be processed to obtain a precise measurement of the target's position relative to the spacecraft. Then, a variety of methods can be used to assess its trajectory and associated risk.\\

The field of trajectory determination is deemed the most appropriate direction for further research based on its current state of the art, and the possibilities for application of established methods. Several techniques have been selected for evaluation: an approach based on the current state-of-the-art in orbit determination, and two systems applying knowledge from other fields: a Kalman Filter, and a novel approach based on artificial neural networks. Based on this decision, an outline is presented for the subsequent work, aiming to deliver its final results near the end of January 2022.