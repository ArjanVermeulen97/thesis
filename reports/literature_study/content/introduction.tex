\chapter{Introduction}
\label{ch:Introduction}

\section{Motivation}

Astronomical phenomena have long fascinated humanity, and few are as awe-inspiring or destructive as a large asteroid impacting Earth. The most well-known example is probably the Chixculub impactor, believed to be the cause of the extinction of the non-avian dinosaurs 66 million years ago. However also more recent examples of impactors destructive potential are known: consider the Tunguska event, flattening over 2000 km$^2$ of forest in 1908, or the 2013 Chelyabinsk meteor, filmed on many dashcams. \\

Considering the seemingly random nature of these events, and their potential for global catastrophe, it is only natural that humanity began its search for these hazardous asteroids. By now, these methods have evolved into automatic systems scanning the entire visible sky every night, resolving targets less than 100 meters in diameter. However, humanity's knowledge of smaller asteroids is still very much incomplete, and observations from Earth are hindered by weather, atmospheric interference, and glare from our Sun. Thus, the need for a space-based system becomes apparent. Of course, astronomy is facing the same juxtaposed challenge and opportunity of massive data streams highlighted in all areas of modern technical research. With the lakes of data that can be generated daily, downlinking the data to Earth and processing it by means of human observers has become impossible; therefore an integral part of our surveys to understand the asteroid population has become its automation.\\

The aim of this literature study, and the proposed follow-up research, is to provide a contribution to our capabilities of autonomous spacecraft assessing the hazard of asteroids on Earth; hopefully warning us, before it is too late.

\section{Research Topics}
As outlined above, the research will focus on several topics. Firstly, it is important to understand the population of asteroids that can actually pose a threat to Earth. This will be discussed in \autoref{ch:population}. Then, a short description of how the optical systems used for imaging work will be given in \autoref{ch:optical}. This is primarily important for the practical side of the research work, as the current state of the field of optics is far beyond our expertise. Before continuing to discussing the state of the research in the relevant fields, a small aside is made to give some examples of the current survey capabilities in \autoref{ch:surveys}. It is believed that this is essential to place the system in context, and to ensure its feasibility. \\

After the earlier, more context-oriented, chapters, the later chapters will focus more on determining the content of the research. Firstly, the state-of-the-art in image processing techniques, both classical and using machine learning applications, will be discussed in \autoref{ch:imageprocessing}. Then, the current and hypothesized methods for determining trajectories of objects, and of asteroids in specific, will be described in \autoref{ch:trajectory}. This review will conclude with a derivation of the applicable knowledge gap, thesis questions, and work breakdown in \autoref{ch:conclusion}.