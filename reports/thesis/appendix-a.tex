\chapter{Function GumbelUnc.mat}

\begin{lstlisting}
%% GumbelUnc
% Quantifictaion of Gumbel extrapolation unceirtainty
%% Parameters
% # |N| = Available data points 
% # |r| = Fits per |N|
% # |CI| = Confidence Interval of extrapolated values: |68| , |95| or
% |99.7| %
% # |normalise| = normalises the axis with: |'on'| , |'xon'| , |'yon'| or
% |'off'|
% # Optional: |mu| = mu of the Sample Gumbel |(default = 10)|
% # Optional: |beta| = beta of the Sample Gumbel |(default = 4)|
%% Syntax
% |[Returnlvlplot] = GumbelUnc(N,r,CI,normalise,mu,beta)|
% 
% |[Returnlvlplot,musigma] = GumbelUnc(N,r,CI,normalise,mu,beta)| 
% 
% |[Returnlvlplot,musigma,SampleGumbel] =
% GumbelUnc(N,r,CI,normalise,mu,beta)|
%% Description
% *|GumbelUnc(N,r,CI,normalise,mu,beta)|* quantifies in a 
% normalised way the uncertainty of extrapolation from 2 towards |N|
% available data points. It plots the return level plot and saves under
% GumbelUnc. Both axis can be normalised with |normalise|. Furthermore the
% sample Gumbel can be adjusted entering for |mu| and |beta|.
%%
% *|[Returnlvlplot] = GumbelUnc(N,r,CI,normalise,mu,beta)|* gives same 
% output as |GumbelUnc(N,r,CI,normalise,mu,beta)|
%%
% *|[Returnlvlplot,musigma] = GumbelUnc(N,r,CI,normalise,mu,beta)|* stores
% the mu and sigmas of |N| normal distributions in the workspace in |musigma|
%%
% *|[Returnlvlplot,musigma,SampleGumbel] =
% GumbelUnc(N,r,CI,normalise,mu,beta)|*
% Gives a figure of the Gumbel from which N samples are drawn for |r| times
% per N. Negative values of the Sample Gumbel or a Sample Gumbel that is
% elevated too high may distort return level plot outcomes. 

%% Function
function [Returnlvlplot,musigma,SampleGumbel] = ...
    GumbelUnc(N,r,CI,normalise,mu,beta)

%some in- and output checks
if nargout > 3
    error('GumbelUnc:Too many outputs, give a maximum of 3 outputs');
end
        

if nargin > 6
    error('GumbelUnc: Too many inputs, give maximum of 6 inputs');
elseif nargin < 4
    error('GumbelUnc: Too fieuw inputs, give a minimum of 4 inputs');
end

switch nargin
    case 4
        mu = 10;
        beta = 4;
    case 5
        error('Specify both mu and beta for sample Gumbel');
end

if mu > N;
    error('mu cannot be greater than N.')
end

if CI == 68;
    Nsigma = 1;
elseif CI == 95;
    Nsigma = 2;
elseif CI == 99.7;
    Nsigma = 3;
else 
    error('Confidence Interval (CI) must be equal to 68%, 95% or 99.7% without percent sign.')
end

%%
% Construction Sample Gumbel
%
% The confidence interval from which is Sampled is between
% 0.1% and 99.9%.
%confidence interval
Rb = mu - (beta * log(-log(0.999)));
Lb = mu - (beta * log(-log(0.001)));

%creating values that are drawn (X) acording to the Sample Gumbel (f)
X = linspace(Lb,Rb,N); 
f = gevpdf(X,0,beta,mu);

%some warnings for Sample Gumbel to prevent distortion in return level plot
if f(N)/max(f) > 0.05;
    warning('Sample Gumbel does not represent extreme values well because of limited use of probabilities. Change mu, beta, Duration0 or V.')
end

if f(find(X >= 0,1))/max(f) >= 0.05
    warning('Sample Gumbel does not represent extreme values well because of significant negative values probabilities.')
end

%% 
% Unceirtanty of extrapolation toward |N| 
%
% Here de data is generated using |polyfit| as fitting tool. Due to the
% logaritmic axis a linear relation exists. The |r| fitted lineair graphs
% are extrapolated towards |FN| which is the |N| th |X| value. A normal 
% distribution of these values is fitted and described by mu and sigma.
% both are stored in the parameter |Normal|. |N| mu's and sigma's are
% iteratively stored in |Allmusigma|.

FN = -log(-log(N/(1+N)));   %the probability of the extremest value

Allmusigma = zeros(N,2);
for i = 2:N;                %minimal of 2 datapoints for fit
   betamu = zeros(r,2);
   Q = zeros(r,1);
   for j = 1:r;             %r extreme value points 
       x = sort(X(discretesample(f,i)));        %Sort values 
       P = -log(-log((1:i)/(1+i)));             %Plot positions
       betamu(j,:) = polyfit(P,x,1);            %Param. of lineair graph
       Q(j,:) = betamu(j,2) + (betamu(j,1)*FN); %All extrapolated values
   end
  Normal = fitdist(Q,'Normal');                 %Fit a Normal distribution 
  Allmusigma(i,:)= [Normal.mu Normal.sigma];    %Store the mu and sigma's 
end

%% 
% Plot normal distributions

%normalisation towards Xtrue
Xtrue = ones(N,1)*(mu+(beta*FN));

NormaliseN = ((1:N)./N)*100;
NormaliseXmu = ((Allmusigma(:,1) - Xtrue)./Xtrue)*100;
NormaliseXup = (((Allmusigma(:,1)+(Nsigma*Allmusigma(:,2)))...
    - Xtrue)./Xtrue)*100;
NormaliseXdown = (((Allmusigma(:,1)-(Nsigma*Allmusigma(:,2)))...
    - Xtrue)./Xtrue)*100;

%Sample Gumbel that represents the data
SampleGumbel = figure;
plot(X,f,'k')             
xlabel('$X$','Interpreter','LaTeX')
ylabel('$f$','Interpreter','LaTeX')
title({'Sample Gumbel'},'Interpreter','LaTeX') 

set(gcf, 'Color', 'none')
set(gcf, 'Units', 'centimeters')
set(gcf, 'OuterPosition', [5, 5, 11.7, 9.5])

%manage output arguments
switch nargout
    case 0
        close figure 1
    case 1
        close figure 1
    case 2
        close figure 1
end

%check axis for normalisation
normalis1 ='off';
normalis0 ='on';
normalis2 ='yon';
normalis3 ='xon';

if strcmp(normalis0,normalise)==1;   
    Returnlvlplot = figure;
    hold on
    plot(NormaliseN(2:N),NormaliseXmu(2:N),'k')
    plot(NormaliseN(2:N),NormaliseXup(2:N),'k')
    plot(NormaliseN(2:N),NormaliseXdown(2:N),'k')
    plot(NormaliseN,zeros(N,1),'--k')
    xlabel('$N_{normalised}~[\%]$','Interpreter','LaTeX')
    ylabel('$X_{normalised}~[\%]$','Interpreter','LaTeX')
    legend({'$\mu$',['$\mu+' num2str(Nsigma) '\sigma$'],...
        ['$\mu-' num2str(Nsigma) '\sigma$'],'Expected $\mu$'},...
        'Interpreter','LaTeX')
    title([ num2str(CI) '\% Confedence Interval of Gumbel extrapolation from 2 toward~' num2str(N) '~data points'],...
        'Interpreter','LaTeX')
    annotation('textbox',[0.5,0.23,0.31,0.15],'String',...
        {'Information Sample Gumbel:',['$N_{sample}~=~$' num2str(N-1)],...
        ['$\beta_{sample}=~$' num2str(beta)],['$\mu_{sample}~=~$' num2str(mu)]},...
        'VerticalAlignment','middle','HorizontalAlignment','center',...
        'FitBoxToText','on','LineWidth',0.1,'Interpreter','LaTeX')
    hold off

elseif strcmp(normalis1,normalise)==1;
    Returnlvlplot = figure;
    hold on
    plot(2:N,Allmusigma(2:N,1),'k')
    plot(2:N,Allmusigma(2:N,1)+(Nsigma*Allmusigma(2:N,2)),'k')
    plot(2:N,Allmusigma(2:N,1)-(Nsigma*Allmusigma(2:N,2)),'k')
    plot(1:N,Xtrue,'--k')
    xlabel('$N~[-]$','Interpreter','LaTeX')
    ylabel('$X~[F]$','Interpreter','LaTeX')
    legend({'$\mu$',['$\mu+' num2str(Nsigma) '\sigma$'],...
        ['$\mu-' num2str(Nsigma) '\sigma$'],'Expected $\mu$'},...
        'Interpreter','LaTeX')
    title([ num2str(CI) '\% Confedence Interval of Gumbel extrapolation from 2 toward~' num2str(N) '~data points'],...
        'Interpreter','LaTeX')
    annotation('textbox',[0.5,0.23,0.31,0.15],...
        'String',{'Information Sample Gumbel:',...
        ['$N_{sample}~=~$' num2str(N-1)],...
        ['$\beta_{sample}=~$' num2str(beta)],...
        ['$\mu_{sample}~=~$' num2str(mu)]},...
        'VerticalAlignment','middle','HorizontalAlignment','center',...
        'FitBoxToText',...
        'on','LineWidth',0.1,'Interpreter','LaTeX')
    hold off
    
elseif strcmp(normalis2,normalise)==1;
    Returnlvlplot = figure;
    hold on
    plot(2:N,NormaliseXmu(2:N),'k')
    plot(2:N,NormaliseXup(2:N),'k')
    plot(2:N,NormaliseXdown(2:N),'k')
    plot(1:N,zeros(N,1),'--k')
    xlabel('$N~[-]$','Interpreter','LaTeX')
    ylabel('$X_{normalised}~[\%]$','Interpreter','LaTeX')
    legend({'$\mu$',['$\mu+' num2str(Nsigma) '\sigma$'],...
        ['$\mu-' num2str(Nsigma) '\sigma$'],'Expected $\mu$'},...
        'Interpreter','LaTeX')
    title([ num2str(CI) '\% Confedence Interval of Gumbel extrapolation from 2 toward~' num2str(N) '~data points'],...
        'Interpreter','LaTeX')
    annotation('textbox',[0.5,0.23,0.31,0.15],'String',...
        {'Information Sample Gumbel:',['$N_{sample}~=~$' num2str(N-1)],...
        ['$\beta_{sample}=~$' num2str(beta)],['$\mu_{sample}~=~$' num2str(mu)]},...
        'VerticalAlignment','middle','HorizontalAlignment','center',...
        'FitBoxToText','on','LineWidth',0.1,'Interpreter','LaTeX')
    hold off
    
elseif strcmp(normalis3,normalise)==1;
    Returnlvlplot = figure;
    hold on
    plot(NormaliseN(2:N),Allmusigma(2:N,1),'k')
    plot(NormaliseN(2:N),Allmusigma(2:N,1)+(Nsigma*Allmusigma(2:N,2)),'k')
    plot(NormaliseN(2:N),Allmusigma(2:N,1)-(Nsigma*Allmusigma(2:N,2)),'k')
    plot(NormaliseN,Xtrue,'--k')
    xlabel('$N_{normalised}~[\%]$','Interpreter','LaTeX')
    ylabel('$X~[F]$','Interpreter','LaTeX')
    legend({'$\mu$',['$\mu+' num2str(Nsigma) '\sigma$'],...
        ['$\mu-' num2str(Nsigma) '\sigma$'],'Expected $\mu$'},...
        'Interpreter','LaTeX')
    title([ num2str(CI) '\% Confedence Interval of Gumbel extrapolation from 2 toward~' num2str(N) '~data points'],...
        'Interpreter','LaTeX')
    annotation('textbox',[0.5,0.23,0.31,0.15],'String',...
        {'Information Sample Gumbel:',['$N_{sample}~=~$' num2str(N-1)],...
        ['$\beta_{sample}=~$' num2str(beta)],['$\mu_{sample}~=~$' num2str(mu)]},...
        'VerticalAlignment','middle','HorizontalAlignment','center',...
        'FitBoxToText','on','LineWidth',0.1,'Interpreter','LaTeX')
    hold off
end

%manage output arguments
musigma = Allmusigma;

switch nargout
    case 0
        clear musigma
    case 1
        clear musigma
end

set(gcf, 'Color', 'none')
set(gcf, 'Units', 'centimeters')
set(gcf, 'OuterPosition', [5, 5, 11.7, 9.5])

end

%% Note
% This function uses *discretesample.m* and thus before use this must
% be present in the same directory


\end{lstlisting}
